%% Date Created: March 4, 2022
%% Available on GitHub at: https://github.com/elcarrillo/academic_cv_latex


%% Add/remove packages and page formatting in preamble.tex
\documentclass[a4paper,10pt]{article}
% Package imports for layout, fonts, icons, colors, etc.
\usepackage[margin=2cm]{geometry} % Page layout
\usepackage{titlesec}             % Custom section titles
\usepackage{setspace}             % Line spacing
\usepackage{enumitem}             % Custom list formats
\usepackage{xcolor}               % Custom colors
\usepackage{fancyhdr}             % Custom headers/footers
\usepackage{hyperref}             % Clickable URLs and links
\usepackage{fontawesome}          % Social media icons (GitHub, email)
\usepackage{tabularx}             % Responsive tables
\usepackage{hanging}              % Hanging indents for citations
\usepackage{helvet}               % Sans-serif font (Helvetica)
\usepackage{datetime}             % Custom date/time

% \usepackage[backend=biber,style=authoryear,maxbibnames=10]{biblatex} % bibtex
% \addbibresource{bib.bib}


% redefine \today to show only month and year
\renewcommand{\today}{\monthname[\the\month] \the\year}

\renewcommand{\familydefault}{\sfdefault} % Set default font to sans-serif
\definecolor{NavyBlue}{RGB}{0,0,128}      % custom colors

% Title formatting for sections
\titleformat{\section} % change heading color here  
  {\color{NavyBlue}\normalfont\Large\bfseries}{\thesection}{1em}{}[{\titlerule[0.8pt]}]
\titleformat{\subsection}  
  {\normalfont\bfseries}{\thesubsection}{1em}{\underline}[]
\titleformat{\subsubsection}  
  {\normalfont\bfseries\itshape}{\thesubsubsection}{1em}{}[{}]

\titlespacing*{\subsubsection}{0pt}{0ex}{0ex}

% line spacing
\setstretch{1.2}

% Footer
\pagestyle{fancy}  % Set the general footer for all pages
\fancyhf{}         % Clear all header and footer fields

%% Define footer for all pages 
\renewcommand{\headrulewidth}{0pt}  % Remove the line at the top
\fancyfoot[L]{[Your Name]}    % Left-aligned name, CHANGE NAME HERE
\fancyfoot[C]{\thepage}       % Center-aligned page number
\fancyfoot[R]{\today}         % Right-aligned date

% Custom footer for the first page
\fancypagestyle{firstpagefooter}{
    \fancyhf{}               % Clear header and footer on the first page
    \fancyfoot[C]{\thepage}  % Custom center footer text
    \fancyfoot[R]{\today}    % Custom right footer text
}
%% To use custom footer, insert the following 
%% command after title text: \thispagestyle{firstpagefooter} 

% custom hyperlink formatting 
\hypersetup{
    colorlinks=true,         % Keep links like URLs visible
    linkcolor=black,         % Link color
    urlcolor=blue,           % URLs color
    citecolor=black,         % Citation links colors
    linkbordercolor={1 1 1}, % Hides boxes around links (for page numbers)
    pdfauthor={Your Name},
    pdftitle={Curriculum Vitae},
    pdfsubject={Academic CV},
    pdfkeywords={Your, Keywords},
    bookmarksopen=true,      % add pdf bookmarks
    bookmarksnumbered=true,
}

%% Custom subheading - can be used in place of subsection heading 
\newcommand{\subheading}[1]{\vspace{0.5cm}\noindent\makebox[\textwidth][l]{\underline{\textbf{#1}}}\par\vspace{0.3cm}} 
    % "makebox" controls overfull underlines

\raggedbottom %% stops latex from stretching content to fill the page


%% Globally remove all section(and subsections) title numbering
\setcounter{secnumdepth}{0}

 % call preamble

%% ===============================================================================
%% =============================== Begin Document ================================
%% ===============================================================================
\begin{document}

%% ======== BEGIN Header for first page ========
\title{\LARGE\bfseries [Your Name]}
\author{}
\date{}

\maketitle \vspace{-2.0cm} %vspace -# gets rid of space from empty \author and \date

% customized title here 
\begin{center}
    [Your Title] $\cdot$ [Your Department] $\cdot$ [Your Institution] \\
    
    \faEnvelope\ \href{mailto:YourEmail@example.com}{YourEmail@example.com} $\cdot$ 
    \faGlobe\ \href{https://your-website.com}{your-website.com} $\cdot$ 
    \faGithub\ \href{https://github.com/your-github}{github.com/your-github} % Optional social icons
\end{center}
%% ======== END Header for first page ========

%% Modify footer in preamble.tex -"Define footer for all pages"   
\thispagestyle{firstpagefooter} %% use custom footer on the first page

%% ======== Begin CV sections ========
\section{Education}
    \begin{tabularx}{\textwidth}{>{\raggedright\arraybackslash}p{2.5cm} X}
    [Year Start] - [Year End] & \textbf{Institution Name}, Location \\
                   & Degree, Field of Study \\
                   & Thesis: \textit{Thesis Title} \\
                   & Advisor: [Advisor Name] \\
\end{tabularx}

\section{Publications}

    \subsection{In Review} 
    \hangindent=1.5em \hangafter=1
    \textbf{Your Name}, Co-authors, (Year). Paper Title. Journal Name.
    
    \subsection{In Preparation}
    \hangindent=1.5em \hangafter=1
    \textbf{Your Name}, Co-authors, (Year). Paper Title. Journal Name. 

\section{Invited Talks}
    \begin{tabularx}{\textwidth}{>{\raggedright\arraybackslash}p{1cm} X}
    Year & \textbf{Talk Title}, Conference/Organization Name \\
    \end{tabularx}

\section{Conference Abstracts}
    \begin{tabularx}{\textwidth}{>{\raggedright\arraybackslash}p{1cm} X}
    Year & \textbf{Your Name}, Co-authors, Conference Title. Conference, Location.
    \end{tabularx}

\section{Honors and Awards}
    \begin{tabularx}{\textwidth}{>{\raggedright\arraybackslash}p{1cm} X}
    Year & \textbf{Award Title}, Awarding Organization.
    \end{tabularx}

\section{Research Interests} %% use comma-separated list for compactness (instead of itemized list)
    [Your research interests in list form]

\section{Research Experience}
    \begin{tabularx}{\textwidth}{>{\raggedright\arraybackslash}p{2.5cm} X}
    Year Start - Year End & \textbf{Role Title}, Organization, Location \\
                          & Description of your responsibilities and research projects.
    \end{tabularx}

\section{Professional Experience}
    \begin{tabularx}{\textwidth}{>{\raggedright\arraybackslash}p{2.5cm} X}
    Year Start - Year End & \textbf{Role Title}, Organization, Location \\
                          & Description of your professional roles.
    \end{tabularx}

\section{Field Experience}
    \begin{tabularx}{\textwidth}{>{\raggedright\arraybackslash}p{1cm} X}
    Year & \textbf{Role Title}, Location \\
         & Description of your fieldwork.
    \end{tabularx}

\section{Teaching and Outreach}
    \subsection*{Organization Name}
    \begin{tabularx}{\textwidth}{>{\raggedright\arraybackslash}p{2.5cm} X}
    Year & \textbf{Role Title}, Description of your responsibilities.
    \end{tabularx}

\section{Training and Professional Development}
    \begin{tabularx}{\textwidth}{>{\raggedright\arraybackslash}p{1cm} X}
    Year & \textbf{Event Title}, Event Location.
    \end{tabularx}

\section{Professional Memberships and Affiliations}
%% use a comma-separated list for compactness (instead of an itemized list)
    List your memberships here.
%% ======== End CV sections ========
\end{document}
%% ===============================================================================
%% =============================== End Document ==================================
%% ===============================================================================




%% =============================== GENERAL FORMATS ===============================

%% SECTION TEMPLATE
%% ------------------------------------
% \section*{Section Title}
% \begin{tabularx}{\textwidth}{>{\raggedright\arraybackslash}p{2.5cm} X}
%   Year & \textbf{Role}, Organization, Location \\
%        & Description of role or project.
% \end{tabularx}
%% ------------------------------------

%% HANGING INDENTS FOR PUBLICATIONS OR ABSTRACTS
%% --------------------------------------------------------------------------------------------
%% Use a hanging indent to improve readability of references and abstract listings:
%%
%% Example:
% \hangindent=1.5em \hangafter=1
% First line of reference or abstract entry.
%
% \hangindent=1.5em \hangafter=1
% \noindent Next reference or abstract entry.
%%
%% Note: Use empty lines (not \\) for new lines between items.

%% ============================= USING BIBLATEX =============================
%% BibLaTeX is ideal for managing long lists of publications, abstracts, etc.

%% STEPS TO USE BIBLATEX:
%% (1) Add entries to a .bib file.
%% (2) In your preamble:
%    \usepackage[backend=biber,style=authoryear,maxbibnames=10]{biblatex}
%    \addbibresource{bib.bib}
%%
%% (3) Decide how to display items:

%% CASE A: ALL ITEMS IN ONE LIST
% \nocite{conferece2024,publication2023}  % Add specific keys or use \nocite{*} for all
% \printbibliography[heading=none]

%% CASE B: SEPARATE BY KEYWORDS
% Use keywords in your .bib entries (e.g., keyword={inreview}).
% \nocite{Carrillo2024_inreview,Ruefer2024_inprep}
% \printbibliography[heading=none,keyword=inreview]

%% (4) CUSTOMIZATION:
% - Add subheadings:
%   \printbibliography[heading=subbibliography,title={In Review},keyword=inreview]
%
% - Include all .bib items without citing individually:
%   \nocite{*} (do this early in the document, before \printbibliography)

%% EXAMPLE FOR CONFERENCE ABSTRACTS:
% \section{Conference Abstracts}
% \nocite{2024_conference}
% \printbibliography[heading=none,keyword=conference]

%% ANOTHER EXAMPLE USING MULTIPLE CATEGORIES:
% \nocite{*}
% \section{Publications}
% \printbibliography[heading=subbibliography,title={In Review},keyword=inreview]
% \printbibliography[heading=subbibliography,title={In Prep},keyword=inprep]
%
% \section{Conference Abstracts}
% \printbibliography[heading=none,keyword=conference]

%% HIGHLIGHTING YOUR NAME:
%% To make your name appear in bold, edit the .bib entry:
% @article{2024_inreview,
%   author  = {{\textbf{LastName, FirstName}} and Author, A.B. and Author, Third},
%   title   = {Your Article Title},
%   year    = {2024},
%   keywords= {inreview},
%   journal = {Journal_Title}
% }
%% Double braces around {\textbf{...}} ensure BibLaTeX treats it as literal text, preserving formatting.

%% ========================== OTHER FORMATTING EXAMPLES ==========================


%% MISC COMMANDS:
%% \addcontentsline{toc}{section}{SECTION_TITLE}  
%%   - Manually adds an entry to the TOC (does not print in the document), useful for navigation or screen readers.

%% \pdfbookmark[2]{SECTION_TITLE}{SECTION_TITLELabel}
%%   - Adds a PDF bookmark for navigation without affecting TOC entries. Useful if skipping heading levels.


%% END OF GUIDELINES
